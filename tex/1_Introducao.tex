Programar é uma tarefa complexa considerando a quantidade de dúvidas e problemas que surgem no processo de desenvolvimento. 
Quando uma erro precisa ser corrigido,  vários desafios podem surgir: uma nova funcionalidade precisa ser adicionada, um código precisa 
ser refeito, um design precisa ser melhorado, um sistema legado precisa ser migrado entre outros e, por isso os programadores estão em 
uma busca constante de respostas e códigos de exemplos que possam auxiliar nessas tarefas.

Nessa perspectiva, surgiram websites que servem como plataformas em que usuários podem realizar perguntas e também responde-las, 
com o intuito de auxiliá-los a encontrarem soluções para seus problemas. Inicialmente surgiram os tradicionais fóruns sem muitas  funcionalidades,
 basicamente os usuários podiam perguntar e responder as perguntas. Atualmente, essas plataformas se desenvolveram e encontramos verdadeiras
comunidade virtuais com diversas funcionalidades: sistemas de TAGS, ranking de reputação, sistemas de votação para perguntas e respostas e  
filtros para consultas detalhadas como é o caso do StackOverFlow \cite{stack} e o GUJ \cite{guj}. Essas aplicações além de facilitar a busca e disseminação 
de informações acerca de desenvolvimento de software estimulam os usuários a se manterem ativos na plataforma através desses rankings 
de reputação e sistema de votação como uma forma de ludificação dos fóruns tradicionais. Os usuários que possuem uma maior pontuação
recebem privilégios na plataforma, como permissão de editar e excluir respostas.

É comum usuários de todo o mundo buscar na internet soluções para esses problemas que surgem no ciclo de vida do software. As dúvidas,
muitas vezes, se repetem e muito dos problemas apresentados revelam  características fundamentalmente semelhantes. Diante disso, na engenharia
 de software surgiram os padrões de projeto (Design Patterns). Segundo Christopher Alexander “Cada padrão descreve um problema que ocorre uma 
e outra vez em nosso ambiente e, em seguida, descreve o núcleo da solução para esse problema, de tal forma que você pode usar essa solução um milhão
de vezes Sem jamais fazê-lo da mesma maneira duas vezes”\cite{gamma1995design}. Embora essa definição do autor fazer referência a construção 
de prédios e cidades ela se aplica adequadamente ao conceito de Design Patterns em linguagens de programação orientada a objeto \cite{alexander1977pattern}. 
Em suma, o Design Patterns na programação descreve uma solução geral para um problema que ocorre com frequência em determinado contexto.

Sua utilização pode melhorar o desempenho do software, tornar seu codigo mais limpo e elegante, facilitar a manutenção e refatoração. Apesar da utilização 
de Design Patterns ser uma boa prática entre os programadores e uma recomendação, sua aplicação não é trivial, exigindo além do conhecimento técnico  
uma experiência para identificar se é pertinente a sua utilização em determinado problema e  qual padrão de projeto utilizar. Daí surge a motivação para a 
criação do projeto deste trabalho, o e-Design Patterns,  um software educacional, como aplicativo para Smartphone nos moldes das comunidades virtuais 
como o stackoverflow e o GUJ, com seu conteúdo focado em Design Patterns.

É notório o impacto que o desenvolvimento da tecnologia vem imprimindo em nossas vidas de diferente formas. A crescente disponibilidade dessas 
tecnologias consequência da redução no custo dos hardwares aliado ao surgimento de softwares cada vez mais aplicados a problemas reais, contribui
 diretamente para o aumento da demanda naa utilização da informática na educação \cite{brandao1998repensando}. 
Não existe uma definição exata de software educacional, pois ainda é uma questão em aberto, mas em uma definição ampla, pode se entender como 
aplicativos que reforcem conteúdos educacionais de forma interativa, auxilie o processo de ensino-aprendizagem e contribua para o enriquecimento
 intelectual\cite{oliveira2001proposta}.


%%%%%%%%%%%%%%%%%%%%%%%%%%%%%%%%%%%%%%%%%%%%%%%%%%%%%%%%%%%%%%%%%%%%%%%%%%%%%%%%
%%%%%%%%%%%%%%%%%%%%%%%%%%%%%%%%%%%%%%%%%%%%%%%%%%%%%%%%%%%%%%%%%%%%%%%%%%%%%%%%
%%%%%%%%%%%%%%%%%%%%%%%%%%%%%%%%%%%%%%%%%%%%%%%%%%%%%%%%%%%%%%%%%%%%%%%%%%%%%%%%
\section{Objetivos}%
Nesta seção serão descritos os objetivos gerais e os objetivos especícos do trabalho.

%%%%%%%%%%%%%%%%%%%%%%%%%%%%%%%%%%%%%%%%%%%%%%%%%%%%%%%%%%%%%%%%%%%%%%%%%%%%%%%%
%%%%%%%%%%%%%%%%%%%%%%%%%%%%%%%%%%%%%%%%%%%%%%%%%%%%%%%%%%%%%%%%%%%%%%%%%%%%%%%%
%%%%%%%%%%%%%%%%%%%%%%%%%%%%%%%%%%%%%%%%%%%%%%%%%%%%%%%%%%%%%%%%%%%%%%%%%%%%%%%%
\subsection{Objetivo Geral}%

Desenvolver um aplicativo para Smartphone no modelo de comunidades virtuais de perguntas e respostas com seu conteúdo focado em Design Patterns tendo como intuito auxiliar e acelerar o processo de aprendizagem dos usuários.

\subsection{Objetivos Específicos}
\begin{itemize}
	\item Identificar e estudar as funcionalidades, estratégias e elementos das plataformas de Q\&A
	\item Prototipação das Telas
	\item Avaliar o aplicatvio desenvolvido com base na ficha de avaliação de software educacional.
	
\end{itemize}


%%%%%%%%%%%%%%%%%%%%%%%%%%%%%%%%%%%%%%%%%%%%%%%%%%%%%%%%%%%%%%%%%%%%%%%%%%%%%%%%
%%%%%%%%%%%%%%%%%%%%%%%%%%%%%%%%%%%%%%%%%%%%%%%%%%%%%%%%%%%%%%%%%%%%%%%%%%%%%%%%
%%%%%%%%%%%%%%%%%%%%%%%%%%%%%%%%%%%%%%%%%%%%%%%%%%%%%%%%%%%%%%%%%%%%%%%%%%%%%%%%
\section{Organização do Texto}%

O restante do trabalho está organizado da seguinte forma:

\begin{enumerate}
	\item O capítulo 2 apresenta uma visão geral sobre Design Patterns, além de descrever alguns dos mais utilizados.
	\item O capítulo 3 trata sobre as plataformas Q\&A, apresentando os conceitos básicos, características e estratégias comumente adotadas nesses ambientes virtuais que garantem seu funcionamento e sucesso.
	\item O capítulo 4 descreve a nossa proposta para criação do aplicativo e-DesignPatterns e todo o processo de prototipação, além das ferramentas utilizadas.
	\item O capítulo 5 apresenta a avaliação do aplicativo com base na ficha de avaliação de software.
	\item O capítulo 6 conclui o trabalho. Além da conclusão também é apresentado os trabalhos futuros para melhorias e continuidade do projeto
\end{enumerate}




%\section{Normas CIC}
% \href{http://monografias.cic.unb.br/dspace/normasGerais.pdf}{Política de Publicação de Monografias e Dissertações no Repositório Digital do CIC}%
% \href{http://monografias.cic.unb.br/dspace/}{Repositório do Departamento de Ciência da Computação da UnB}

% \href{http://bdm.bce.unb.br/}{Biblioteca Digital de Monografias de Graduação e Especialização}