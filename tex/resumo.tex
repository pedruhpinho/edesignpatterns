Q\&A plataforms, ou plataformas de perguntas e repostas são aplicações muito populares atualmente que permitem seus usuários compartilharem o próprio conhecimento através de perguntas e respostas. Essas plataformas servem como uma fonte online para que usuário encontrem de maneira rápida e eficiente respostas objetivas para suas dúvidas. Muitas aplicações desse tipo tem surgido com o seu conteúdo focado para programação. Dúvidas respondidas e códigos de exemplo ajudam programadores a solucionarem problemas e obstáculos no dia dia de desenvolvedor.
Design Patterns ou Padrões de Projetos são soluções gerais aplicadas na programação orientada a objeto que visam solucionar problemas recorrentes em determinado contexto no projeto de softwares. Um design pattern não é um código pronto para se aproveitar em uma aplicação e sim um modelo para resolver um problema sendo soluções que já foram utilizadas e testadas durante o tempo o que nos dá confiança em sua eficácia. A utilização desses padrões podem trazem mutios benefícios para o projeto de software, como facilitar a manutenção, melhorar a documentação e tornar o software reutilizável. O grande problema para utilização de design Patterns é que na maioria das vezes é necessário ser um desenvolvedor experiente para aplicação correta de determinado padrão.
Visando a popularidade e a eficiência das plataformas Sociais de Q\&A auxiliando programadores com suas dúvidas  e a dificuldade na utilização de desgin patterns por programadores não tão experientes, propomos a criação de um aplicativo para smartphone, no modelo das plataformas Q\&A, o e-Design Patterns. Um aplicativo com intuito de auxiliar e acelerar o processo de aprendizado para utilização de Design Patterns utilizando as estratégias e características dessas plataformas para um aplicativo de sucesso.