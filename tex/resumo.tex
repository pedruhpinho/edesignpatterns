Q\&A platforms, ou plataformas de perguntas e respostas são aplicações muito populares atualmente que permitem seus usuários compartilharem
 o próprio conhecimento através de perguntas e respostas. Essas plataformas servem como uma fonte online para que usuário encontrem de maneira
 rápida e eficiente respostas objetivas para suas dúvidas. Muitas aplicações desse tipo tem surgido com o seu conteúdo focado para a programação.
 Dúvidas respondidas e códigos de exemplo ajudam programadores a solucionarem problemas e obstáculos no dia a dia de desenvolvedor.

Design Patterns ou Padrões de Projetos são soluções gerais aplicadas na programação orientada a objeto que visam solucionar problemas recorrentes
 em determinado contexto no projeto de softwares. Um Design Pattern não é um código pronto para se aproveitar em uma aplicação e sim um modelo
 para resolver um problema sendo soluções que já foram utilizadas e testadas durante o tempo que nos dá confiança em sua eficácia. A utilização desses
 padrões trazem mutios benefícios para o projeto de software, como facilitar a manutenção, melhorar a documentação, tornar o software reutilizável
 entre outros. O grande problema para utilização de Design Patterns é que na maioria das vezes é necessário ser um desenvolvedor experiente para
 aplicação correta de determinado padrão.

Visando a popularidade e a eficiência das Plataformas Sociais de Perguntas e Respostas em auxiliar programadores com suas dúvidas e dificuldades na utilização de
 Desgin Patterns por programadores, propomos a criação de um aplicativo para smartphone, no modelo das Q\&A Platforms,
 o e-DesignPatterns. Um aplicativo com objetivo de auxiliar e acelerar o processo de aprendizagem para utilização de Design Patterns utilizando as estratégias
 e características dessas plataformas para criar um aplicativo de sucesso.