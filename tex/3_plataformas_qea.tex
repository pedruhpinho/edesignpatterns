\section{Breve Histórico}%

O termo web 2.0 emergiu em 2004 e foi popularizado pela empresa americana O'Reilly Media para designar uma nova geração de aplicações e serviços para internet
 que surgiam para formar a "web participativa", exemplos desses serviços e aplicações são  blogs, wikis, redes sociais etc citar \cite{madden2006riding}.  Nesse sentido,
 o desenvolvimento da web criou uma revolução na internet onde  seus usuários passaram de consumidores passivos  para produtores e compartilhadores de conteúdo. 
Cada vez mais usuários utilizam fontes online para encontrar as informações que desejam. Motores de pesquisas, como google, yahoo, ask frequentemente são as portas
 para o encontro dessas informações. Apesar desses mecanismos dominarem as buscas por informações na internet outras fontes de pesquisa e compartilhamento de informações
 surgiram e ganharam força, são os site de perguntas e respostas ou as plataformas sociais de perguntas e respostas \cite{shah2008explorin}.

Q\&A (Questions and Answers) platforms, ou plataformas de perguntas e respostas são ambientes virtuais em que usuários voluntariamente postam perguntas e outros usuarios respondem.
 A plataforma armazena em seus bancos de dados esse rico conteúdo composto pelas perguntas e respostas e disponibiliza para que os usuários possam pesquisá-lo. Esse processo torna
 esses ambientes virtuais em sistemas auto-sustentáveis, ou seja, são os próprios usuários que geram o conteúdo disponível. Nesse formato, usuários quando possuem alguma dúvida, 
 ao invés de postarem uma pergunta e terem que aguardar uma respostas, podem primeiramente pesquisar na plataforma se sua dúvida já foi postada por outro usuário e inclusive respondida,
 economizando dessa forma seu tempo.
 
As plataformas de perguntas e respostas vão alem de um local onde perguntas são respondidas, os usuários além de compartilharem seu próprio conhecimento, compartilham experiências e opiniões. De maneira geral, é responsabilidade dos
 própiros questionadores avaliarem as respostas dos outros usuários e filtrarem as informações que melhor satisfazem suas necessidades \cite{kim2007best}. 
 A melhor resposta, avaliada pelo autor da pergunta, fica em destaque e marca o tópico da pergunta como solucionado.

\section{Estratégias e características}%

 Essas plataformas possuem uma estratégia de pesquisa diferente e são mais específicas do que os motores de buscas tradicionais, enquanto o google, por exemplo, busca referências de todos o
 tipos e em toda a web, essas plataformas disponibilizam respostas personalizadas para perguntas indiviuais pesquisadas ou postadas pelos próprios usuários. Em motores de buscas como Yahoo!,
 Google e Bing os usuários recebem como resposta uma lista de sites e precisam garimpar entre eles as informações que consideram relevantes. Já nas plataforemas de perguntas e respostas os usuários apenas precisam
 escolher, em sua concepção, a melhor resposta para a pergunta que postou ou pesquisou.

Como característica da web 2.0, esses sites ganharam um carater social, ou seja, essa plataformas que no início eram utilizadas apenas como ponto para compartilhamento de informações (fóruns tradicionais)
com usuários postando perguntas e respostas, decidiram  introduzir páginas para cadastramento de um perfil completo dos usuários, onde eles podem registrar diversas informações pessoais, como 
nome completo, idade, redes sociais, preferências de estudo, entre outras informações. Assim os usuários podem conhecer um pouco mais uns aos outros. Alem disso, como uma forma de aumentar a comunicação e interação
entre os usuários, essas plataformas adicionaram chats como outra funcionalidade. Os chats permitem uma maior interação entre os usuários, quando o contato através do topicos de perguntas e respostas não é suficiente.

Existem vários sites que adotam esse tipo de plataforma, alguns deles são Yahoo! Respostas, Amazon’s Askville, Quora e Mind the Book. A propria Google, que é a lider em busca de conteúdo online, também possui um serviço de Q\&A, 
o Google Answres, que falhou e foi descontinuado. Muitas razões foram levantadas para justificar esse insucesso. Uma dos principais fatores que influenciam para o sucesso de uma plataforma
 de perguntas e respostas é a participação do usuários \cite{shah2008explorin}. Constatando a importância desse fator, essas plataformas começaram adotar estratégias que motivassem e tornassem cada vez mais agradável a particiapação dos usuários. Desta forma, com a participação constante dos usuários, o sistema, como citado no inicio do capítulo, se torna auto sustentável e sua chance de sucesso é maior. Segue algumas estratégias de gamificação adotadas para  motivar a particiapção dos usuários:

\begin{itemize}
	\item Reputação através de pontuação: Usuários recebem pontos por sua participação e pela qualidade dessa participação. Quanto maior sua sua pontuação, maior sua reputação no site. 
	\item Privilégios: Usuários recebem privilégios de acordo com alguns fatores, que podem variar, mas são adiquiridos principalmente com base na reputação. Exemplos desses privilégios são: Permissões para editar e excluir perguntas e respostas.
	\item Badges (Medalhas): Além da pontuação recebida, usuários também recebem medalhas como reconhecimento a sua participação no site.  
\end{itemize}

Como facilitador para utilização do site, essas plataformas contam com serviço de busca eficiente. Além de filtros para realizar pesquisas persoalizadas, adotam também sistemas de TAG's. Todas perguntas são marcadas com suas respectivas áreas.
TAG'S são marcadores utilizados pelos usuários em suas perguntas. Uma pergunta sobre como implementar um web-service rest, por exemplo, pode ser marcada com TAG's como: java, web-service e JBoss. Essas TAG's são temas que tem haver com a implementação de 
um Web-service REST. A marcação com TAG's facilita para os usuários a procura por conteúdos específicos, ou seja, clicando em determinada TAG topicos marcados com a mesma TAG serão apresentados.

\section{StackExchange}%

No início, essas plataformas abordavam todos os tipos de conteúdos, ou seja, os usuários compartilhavam informações sobre qualquer assunto. Após algum tempo, viu-se a necessidade de criar plataformas no mesmo seguimento, com as mesmas
 funcionalidades, porém mais específicas. Novos Q\&A sites surgiram, focando em conteúdos específicos, como matemática, esportes, programação etc. Um exemplo muito interessante, de extremo sucesso e que serviu como base para o desenvolvimento
 desse trabalho é o StackExchange.

O StackExchange é uma plataforma voltada para unir pessoas com o mesmo interesse. Ela é subdividida em comunidades, cada um com o seu foco específico em determinado assunto, onde cada comunidade dessa possui seu próprio domínio e plataforma de perguntas e respostas. Cada comundiade dessa é autosuficiente, no sentido em que cada uma delas possui sua própria base usuários, administradores e moderadores, regras, e base de dados baseado no assunto tratado. Esses dados não estão necessariamente atrelados ao StackExchange, que atua como gerenciador maior dessas plataformas, dando suporte a correção de bugs e melhorias na plataforma, usada pelas comunidades.


\subsection{StackOverflow}%

O StackOverflow é uma comunidade de Perguntas e Respostas do StackExchange, que tem seu conteúdo voltado exclusivamente para a programação e faz parte de uma rede de sites desse tipo chamado Stack Exchange. O StackOverFlow é lider na web quando se trata de sites Q\&A voltados para programação.
Ele conta com uma base solida de usuários que participam ativamente do site, imersos pelas estratégias de gamificação e que mantém o site atualizado. Além disso, diferentes das outra plataformas, o site possui um ambiente focado em perguntas e repostas objetivas,
conteúdos considerados não objetivos, são editados e podem ser até excluídos. Esse é um dos principais fatores para o sucesso da plataforma, pois condiciona o site  a ter em seu banco de dados um conteúdo de perguntas e respostas totamente objetivo o que facilita
 a busca por informações. 


\subsection{API}%
A plataforma possui sua própria API (Application Provider Interface), que possibilita integrações de outros sistemas com o StackExchange, onde se tem acesso a informações muito interessantes. No escopo do nosso trabalho, pensamos em validar as respostas dos nossos usuários conforme sua reputação no StackOverflow, adquirindo uma credibilidade inicial para montarmos nossa própria base de dados e de usuários.