Conforme abordado no Capítulo 4, o uso de design Patterns por programadores em determinado contexto no projeto de softwares não é tarefa trivial e, na maioria das vezes, requer um certo nível de experiência. Além de conhecer os padrões disponíveis, os desenvolvedores devem saber identificar em qual contexto aquela solução se aplica. O aplicativo e-Design Patterns foi idealizado justamente para minimizar esses requisitos de experiência, buscando facilitar o entendimento sobre o conceito de DesignPatterns, os padrões disponíveis e e como aplica-los em seus projetos. O e-Design Patterns é um protótipo de aplicativo para smartphone, tablets e browser no molde das plataformas sociais de Perguntas e Respostas.

\section{ Motivação}
Desenvolvedores enfrentam uma problemática prática em muitos de seus dias de trabalho. Seja por falta de experiência, ou por estarem enfrentando desafios novos, muitas vezes não é claro qual padrão de projeto é mais adequado à solução do problema em vigor.

Em nossa pesquisa, não encontramos um aplicativo ou sistema parecido com o que propomos. Existem aplicativos que atuam como referência ao conteúdo de padrões de projeto, porém não fornecem nenhuma interatividade com a pessoa interessada, e o conteúdo muitas vezes se mostra denso e nada prático para aplicar na rotina corrida de um desenvolvedor.

Buscamos fornecer com o nosso aplicativo um ambiente onde o usuário possa consultar sucintamente o conteúdo que desejar e interagir com o sistema de forma inteligente e perguntar a usuários mais experientes e tirar dúvidas específicas.

\section{Descrição do Aplicativo}
O projeto do nosso aplicativo visa resolver essa carência descrita acima, e mudar o jeito como programadores e estudantes de computação interagem com essa temática dos design patterns. Desejamos que seja fácil reconhecer e aplicar os padrões para o seu caso de uso, ajudando a comunidade a encontrar as melhores soluções pros seus contextos específicos.

\subsection{Core Features}

Definiremos aqui as principais funcionalidades para a plataforma eDesignPatterns funcionar do jeito que foi concebida. Realizamos, com orientação da Profa. Dra. Letícia, que deveríamos nos atentar como a plataforma se tornaria um ecosistema sustentável, com garantia da qualidade do conteúdo alimentado pelos usuários, e que motivasse a volta dos usuários.

Com essas três funcionalidades, concebidas durante a concepção do nosso projeto, teríamos uma base funcional para a plataforma, que nos proveria interação entre os usuários, organização dos conteúdos por assuntos e interesses do usuário, e uma espécie de gamificação através da reputação dos usuários.

\subsubsection{Platafoma Q\&A}
Como vimos no Capítulo 3, uma grande tendência da web participativa são as  plataformas de pergunta e resposta. São uma poderosa forma de organizar pessoas e organizar o conteúdo produzido por elas.  
\subsubsection{Tags}

\subsubsection{Reputação}

\subsection{Features desejáveis}


\section{Processo de Software}

Em um projeto de software, é necessário seguir um conjunto de passos, que servem como um roteiro para direcionar o rumo do desenvolvimento e para que o resultado tenha alta qualidade. Segundo Pressman \cite{pressman2016engenharia}, o processo de software é uma metodologia para as atividades, ações e tarefas necessárias para desenvolver um software de alta qualidade. Existem alguns desse modelos e a escolha de um deve ser feita de acordo com o software a ser desenvolvido e os requisitos levantados, mas é certo que todos o modelos oferecem ao projeto de software em diferentes níveis, estabilidade, controle e organização.

\subsection{Modelo Evolucionário} 

O modelo de processo evolucionário, é uma abordagem interativa, onde é possível desenvolver versões de software cada vez mais completas com base em versões iniciais. Essa abordagem é muito utilizada, pois permite o desenvolvimento do software  mesmo com todas as mudanças que podem ocorrer nos requisitos, regras de negócio, funcionalidades etc. Essas mudanças durante todo o processo impedem que o desenvolvimento siga um planejamento linear, por isso o modelo evolucionário permite a criação de versões do software com base no que já foi estabelecido e, essas versões servem como ponto de partida para a continuação do software até o produto final.Existem dois modelos comum em processos evolucionários, a Prototipação, que foi o modelo escolhido para a construção do e-DesignPatterns e, o modelo espiral \cite{pressman2016engenharia}. 


\begin{itemize}
	\item Prototipação: Considerando o aumento nos custos para o desenvolvimento de softwares e o crescente número de sistemas construídos que falharam em satisfazer as necessidades dos clientes, as organizaçõe estão utilizando cada vez mais a prototipação. A prototipação é  uma implementação parcial de um sistema que serve para que clientes e desenvolvedores aprendam mais sobre o produto final, ou seja, possibilita uma melhor compreensão das necessidades que devem ser atendidas. Um protótipo permite que clientes avaliem o que já foi desenvolvido até o momento e forneçam um retorno (feedback) que servirá para aprimorar os requisitos até que seja alcançado o software final \cite{davis1992operational}. 
\end{itemize}

\section{Prototipação do e-DesignPatterns}

O processo de prototipação do e-DesignPatterns passou por 4 etapas:

\begin{enumerate}
	\item Brainstorming: Nessa primeira etapa o grupo de desenvolvimento se reuniu e fez o leventamento de todas as idéias para o software. Além do levantamento de requisitos, foram tratadas questões de design, estratégias de implementação, delagação de tarefas, roteiro de desenvovimentos, metas, público alvo e ferramentas a serem utilizadas. 
	\item Produção de Wireframes: Na segunda etapa, todas a idéias levantadas no brainstorming formam utilizadas na produção dos wireframes, ou seja, esboços das principais telas do aplicativo que visam estabelecer as melhores estratégias para que a experiência do usuário com o software (UX) sejam as melhores possíveis: usabilidade, acessibilidade e prazer proporcionado na interação entre o usuário e o produto. O grupo optou em realizar o desenvolvimento dos wireframes em papel como estratégia de acelerar o processo e não ter que utilizar mais um software de apoio. Não foram criados wireframes para todas as telas, somente para as principais. Como estratégia de poupar tempo, os wireframes foram desenvovidos somente para as telas que ofereciam um maior desafio.
	\item Design das telas: Nesta etapa, com base nos wireframes desenvolvidos e nas estratégias estabelecidas, utilizamos a ferramenta Figma para design final de todas a telas, inclusive aquelas que não tiveram wireframes relacionados. 
	\item Interação entre as telas: Na quarta e última etapa, utilizamos uma ferramenta de design como o figma, mas que também permite criar interações entre as telas, o Marvel. As telas desenvovidas no Figma, na 3 terceira etapa, foram exportadas para arquivos no formato .png e em seguida foram importadas para o Marvel, que foi utilizado para criar as devidas interações entre as telas.
\end{enumerate}


\section{Ferramentas Utilizadas}

As seguintes ferramentas forma utilizadas no projeto do e-DesignPatterns, tanto para a prototipação do aplicativo como para comunicação e organização.

\subsection{Slack}
Slack \cite{slack} é um ferramenta colaborativa que permite a comunicação entre usuários de um grupo (time) de maneira fácil e interessante. Para utilizar a plataforma basta criar um time e convidar outros participantes para compo-lo. Alem da comunicação por mensagens em tempo real a ferramenta também permite a criação de canais que podem ser utilizados para que os usuários do time tratem de assuntos diferentes. Diversas outras funcionalidades são oferecidas pela plataforma, uma muito interessante é a possibilidade da integração com outras ferramentas, como o gitHub e o Marvel.  A ferramenta também oferece notificações instantâneas sobre atividades realizadas no grupo, permite mensagens privadas, compartilhamento de arquivos entre outros funcionalidades.

\subsection{Git e GitHub}

Git \cite{git} é um sistema de controle de versões distribuídas e gerenciamento de código, com ênfase em velocidade. É a ferramenta lider no mercado, pois é extremamente eficiente e robusta. Para utilizá-lo é preciso criar um repositório local ou remoto onde os arquivos serão armazenados e versionados de acordo com cada utilização. Para este trabalho, escolhemos utilizar um repositório remoto para que todos os integrantes do grupo tivessem acesso. O repositório escolhido foi o GitHub, também muito eficiente e de larga utilização A opção por um repositório remoto foi fetia para que todos os componentes do grupo tivessem acesso. Nesse repositório armazenamos os arquivos latex do trabalho escrito.

\subsection{Figma}

Figma \cite{figma} é uma ferramenta online de design de interface e que permite colaboração em tempo real. A ferramenta disponibiliza a criação de times para o desenvolvimento colaborativo de interfaces, oferece controle de versões e designs responsivos entre várias outras funcionalidades.


\subsection{Marvel}

Marvel \cite{marvel} é um ferramenta colaborativa de design e prototipação e está disponível para web e smartphones. Oferece ferramentas para criação de telas mas também permite importar imagens do Sketch ou Photoshop, além de permitir a sincronização com cloud storages. Sua simples interface de edição permite lincar todos os designs juntos e adicionar gesture e transiçoes para tornar o seu prototipo fiel com um aplicativo real ou um website.

\section{Telas e suas funcionalidades}